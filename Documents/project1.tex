% !TEX TS-program = pdflatex
% !TEX encoding = UTF-8 Unicode

% This is a simple template for a LaTeX document using the "article" class.
% See "book", "report", "letter" for other types of document.

\documentclass[11pt]{article} % use larger type; default would be 10pt

\usepackage[utf8]{inputenc} % set input encoding (not needed with XeLaTeX)
\usepackage{hyperref}
%%% Examples of Article customizations
% These packages are optional, depending whether you want the features they provide.
% See the LaTeX Companion or other references for full information.

%%% PAGE DIMENSIONS
\usepackage{geometry} % to change the page dimensions
\geometry{a4paper} % or letterpaper (US) or a5paper or....
% \geometry{margin=2in} % for example, change the margins to 2 inches all round
% \geometry{landscape} % set up the page for landscape
%   read geometry.pdf for detailed page layout information

\usepackage{graphicx} % support the \includegraphics command and options

% \usepackage[parfill]{parskip} % Activate to begin paragraphs with an empty line rather than an indent

%%% PACKAGES
\usepackage{booktabs} % for much better looking tables
\usepackage{array} % for better arrays (eg matrices) in maths
\usepackage{paralist} % very flexible & customisable lists (eg. enumerate/itemize, etc.)
\usepackage{verbatim} % adds environment for commenting out blocks of text & for better verbatim
\usepackage{subfig} % make it possible to include more than one captioned figure/table in a single float
% These packages are all incorporated in the memoir class to one degree or another...

%%% HEADERS & FOOTERS
\usepackage{fancyhdr} % This should be set AFTER setting up the page geometry
\pagestyle{fancy} % options: empty , plain , fancy
\renewcommand{\headrulewidth}{0pt} % customise the layout...
\lhead{}\chead{}\rhead{}
\lfoot{}\cfoot{\thepage}\rfoot{}

%%% SECTION TITLE APPEARANCE
\usepackage{sectsty}
\allsectionsfont{\sffamily\mdseries\upshape} % (See the fntguide.pdf for font help)
% (This matches ConTeXt defaults)

%%% ToC (table of contents) APPEARANCE
\usepackage[nottoc,notlof,notlot]{tocbibind} % Put the bibliography in the ToC
\usepackage[titles,subfigure]{tocloft} % Alter the style of the Table of Contents
\renewcommand{\cftsecfont}{\rmfamily\mdseries\upshape}
\renewcommand{\cftsecpagefont}{\rmfamily\mdseries\upshape} % No bold!

%%% END Article customizations

%%% The "real" document content comes below...

\title{Project 1 Part 1}
\author{Joshua Westbrook \& Arifur Rahman}
\date{} % Activate to display a given date or no date (if empty),
         % otherwise the current date is printed 

\begin{document}
\maketitle

\section{The maps}

The map files are available at \url{https://github.com/Joshua-Westbrook/AIproject1maps}

They are also in the folder that opens after unzipping the .zip file attached and importing it into Eclipse as an existing project.

\section{The algorithm}

The algorimths can be found in in the searches package. The UCS algorithm is fully implemented in SearchAlgo.java but uses UCS.java so it has a specific name. A* and Weighted A* overwrite the h(n) being zero in UCS with them using either the heuristic from DistanceHeuristic.java or the heuristic times the weight. Unforunately the algorithms do not seem to be working fully properly. There is no error where Weighted A* sometimes returning a cheaper path than A*. We have not been able to figure out the cause yet.

\section{Optimizations}
No optimizations have been done beyond that of following the UCS/A*/Weighted A* rules.

\section{Heuristics}
The Heuristic we finally used is the pythagorean formula. We get the best path length from start to goal and from currNode to goal by giving one side of the triangle as the difference in rows, the other as the difference in columns, and getting that square root. We then times that distance between the currNode and goal by .25 since in a best case scenario the distance will be a river of normal to travel between the current node and the goal which has a travel time of .25. This makes the heuristic admissable because it is impossible for the actual travel distance to be less. We then divide the found best case scenario travel time by the best case scenario travel time from the start to the goal in a situation where the start and the goal are connected by a river.

\end{document}
